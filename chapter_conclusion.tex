\chapter{Conclusions}

Knowledge graphs are well suited to represent human knowledge and provide a platform for research applications.  Assembling and querying biomedical knowledge graphs for pharmaceutical discovery applications requires strong foundations in numerous areas including semantics and ontology, cheminformatics, bioinformatics, and other domain areas, software engineering, and data science.
Recent scientific and technological advances have informed and empowered these inquiries in profound ways. By incorporating measures of confidence and relevance, knowledge graphs can empower and augment human cognition for scientific discovery. Badapple, KGAP and TIGA illustrate and exemplify these themes in related but distinct and complementary ways. The other projects described also illustrate and exemplify key aspects of these themes.
