\chapter{Conclusions}

Knowledge graphs are well suited to represent human knowledge and provide a platform for research applications.  Assembling and querying biomedical knowledge graphs for pharmaceutical discovery applications requires strong foundations in numerous areas including semantics and ontology, cheminformatics, bioinformatics, and other domain areas, software engineering, and data science. By incorporating measures of confidence and relevance, with strong semantics informed by domain expertise, knowledge graphs can empower and augment human cognition for scientific discovery. Badapple, TIGA and KGAP illustrate and exemplify these themes in related but distinct and complementary ways. The other projects described, CARLSBAD, OPDDR, and TIN-X, also illustrate and exemplify key aspects of these themes. In each project, biomedical entities and relationships are defined with precision and accuracy, preferably and usually via community-standard formal ontologies.

Ongoing and recent scientific and technological advances have informed and empowered these inquiries in profound ways. Powerful and reliable graph database products including Neo4j, Dgraph, JanusGraph, and others, have been developed and adopted widely for a variety of domains and use cases, several with free community editions, and vibrant user and developer communities accelerating progress. As is typically true in sciences dependent upon advancing technologies, much of the research described in this dissertation could be accomplished more effectively with newer software tools, with larger datasets and faster computation, and these trends should continue. However, the need to evaluate evidence, for confidence and relevance, by data aggregation, semantic harmonization, and domain expertise, as illustrated in this dissertation, should remain fundamental to biomedical data science.
