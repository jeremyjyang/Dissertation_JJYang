\begin{table}
\caption{Data science terms: Visualization}
\label{appendix:glossary_datascience_vis}
\begin{tabular}{p{0.3\linewidth}p{0.6\linewidth}}
\hline
\makecell[r]{\textbf{Heatmap}} & Represents a real valued matrix with a color spectrum.\\
\makecell[r]{\textbf{Histogram}} & Represents the sample distribution of a numerical variable, binned into intervals, shown as bars of height proportional to count in the bin.\\
\makecell[r]{\textbf{Bar Chart}} & Heights of bars proportional to some quantity, more general than histogram as the bars may be categorical (e.g. national populations).\\
\makecell[r]{\textbf{Radar Chart}} & Similar to bar chart, but radial.\\
\makecell[r]{\textbf{Scatter Plot}} & Points on 2D (or maybe 3D) axes.\\
\makecell[r]{\textbf{Box Plot}} & Visualizes the distribution of a variable, including the median, 1st and 3rd quartile.  WIth a "box and whiskers" plot, the whiskers may
indicate the 2nd and 98th percentiles.\\
\makecell[r]{\textbf{Visual Channels}} & Informative characteristics, such as color or shape or size.\\
\makecell[r]{\textbf{Pre-Cognitive Awareness}} & Visualization can facilitate cognition by employing this natural pattern recognition capability.  For example, seeing patches of
color in a heatmap is easier than reading and abstracting from all the numbers.\\
\hline
\end{tabular}
\end{table}
