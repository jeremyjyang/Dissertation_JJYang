
\begin{singlespace}
\begin{longtable}{p{0.3\linewidth}p{0.7\linewidth}}
\caption{Data science terms: Statistics}\\
\multicolumn{2}{c}{\emph{(See statistics textbooks and other resources for more authoritative definitions of these terms.)}}\\ 
\hline
\makecell[r]{\textbf{Confidence Interval (CI)}} & Typical 95\% CI means given the sample data and assumptions, the parameter will be within the CI 95\% of the time if the data is re-sampled.\\
\makecell[r]{\textbf{P-Value}} & Probability of the sample data if the null hypothesis is true.\\
\makecell[r]{\textbf{Probability Distribution}} & For a discrete random variable, gives the probability of any value, for a continuous random variable, gives the probability of any interval.\\
\makecell[r]{\textbf{Statistic}} & Any formulaic numerical descriptor of a sample (e.g. mean, median).\\
\makecell[r]{\textbf{Parameter}} & The parameters of a normal or Gaussian distribution are the mean and variance.  For any such family, the function can be fully defined by the
parameters.  Parametric statistics is largely focused on estimating population (global) parameters and their uncertainty from sample data.\\
\makecell[r]{\textbf{Multiple Testing}} & If a coin is flipped 5 times, all heads, the coin is likely unfair.  However, if 100 coins are all flipped 5 times, it is highly likely at least one will be all heads, and that will provide no evidence of unfairness.  In this way the interpretation of P-values and other metrics much be adjusted in multiple testing scenarios.\\
\makecell[r]{\textbf{Chance Correlation}} & With random variables, correlation may reflect explanatory truths, or simply be due to random chance.\\
\makecell[r]{\textbf{False Discovery Rate}} & A quantification of the probability of false discovery, or chance correlation. (See XKCD comic on M\&M associated risks.)\\
\makecell[r]{\textbf{Confounder}} & When associating two variables, a confounder is another third variable which may be more explanatory.  In an ideally executed randomized controlled trial, confounders are impossible since only the controlled independent variables differ between the control and test groups.\\
\makecell[r]{\textbf{Independent Variables}} & Random variables which are not correlated, not affecting each other or each affected by another variable.\\
\makecell[r]{\textbf{Correlation versus}\\ \textbf{causation}} & Despite being a fundamental truth of statistics and rationality, an common area of confusion and misinterpretation in science, and in human cognition more generally.\\
\makecell[r]{\textbf{Random sampling}} & Selecting a representative subset of a population by some random process.\\
\makecell[r]{\textbf{Sampling bias}} & Non-random sampling.  For example, online surveys are biased toward persons who are online and like to respond to surveys, thus not
representative of the general population.\\
\makecell[r]{\textbf{Population}} & Conceptually the global set of entities of interest.  For analytics pertaining to US federal elections, the population would be all US voters.
For anti-depressant drug trials, the population might be all humans diagnosed with major depressive disorder. In a study, a sample is used to make inferences about
the population.\\
\makecell[r]{\textbf{Frequentist versus}\\ \textbf{Bayesian Statistics}} & Among statisticians and statistical scholarship, there are two main schools, frequentist and Bayesian. Oversimplifying somewhat, these schools don't disagree about math, including Bayes Theorem, but do disagree about interpretations.  Frequentists tend to view population parameters as truth.  Bayesians tend to view evidence as updating belief.\\
\makecell[r]{\textbf{Bayesian Approach}} & Somewhat beyond the basic truth of Bayes Theorem, a Bayesian approach is often iterative, and can adjust the parameters of a distribution, as a way of continuous learning from streams of evidence.\\
\makecell[r]{\textbf{Prior Distribution}} & A Bayesian learning iteration begins with belief in a prior distribution, to be adjusted according to the evidence in the sample data.
In a simple univariate case, this involves multiplication by a Bayesian factor.  For a simple coin flip, we would probably believe in a prior of 50-50\% head vs
tail.\\
\makecell[r]{\textbf{Posterior Distribution}} &	After adjustment of the belief represented by the prior distribution, the result of the Bayesian learning iteration is the posterior distribution.  After flipping our coin and getting 5 heads in a row, our posterior might be 70-30\% head vs tail.\\
\makecell[r]{\textbf{Probability versus}\\ \textbf{likelihood}} & Probability refers to an uncertain outcome modeled by a random variable and probability distribution. Likelihood, in parametric statistics, refers to a parameter, modeled as unvarying, which given a probability distribution and sample data, may be estimated by maximum likelihood estimator (MLE).\\
\hline
\end{longtable}
\end{singlespace}

